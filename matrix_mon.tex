In this appendix, we prove matrix monotonicity of the square root function. We begin by defining matrix monotonicity 
Let $\cD \subset \bbmR$ and $f: \cD \rightarrow \bbmR$. Let $\hat{f}_n: \cA_n(\cD) \rightarrow \bbmH_n$ 
the matrix function generated by $f$ according to (...). 
\begin{definition} \index{matrix monotone function} 
$f: \cD \rightarrow \bbmR$ is \emph{matrix monotone (matrix monotonically increasing)}, if for each $n \in \bbmN$ the function $\hat{f}_n$ 
fullfills 
\begin{align*}
\forall A,B \in \cA_n(\cD): \ A \leq B \ \Rightarrow \ \hat{f}_n(A) \leq \hat{f}_n(B).
\end{align*}
\end{definition}
The concept of matrix monotonicity goes beyond the ordinary monotonicity of a real function, monotonicity as a real function does not imply matrix monotonicity. 
\begin{example}
 The function $t \mapsto t^2$ is not matrix monotone on $[0,\infty]$. 
\end{example}
\begin{lemma} \label{prop:sqrt_matrix_monotonicity}
The function $t \mapsto t^{\tfrac{1}{2}}$ is matrix monotone.
\end{lemma}	
To prove Lemma \ref{prop:sqrt_matrix_monotonicity}, we need some supporting claims we prove first. 

