In this lecture we consider the "asymmetric quantum hypothesis testing" problem. Assume, an experimenter is confronted with a source which emits pairwise independent and equally prepared quantum systems. Given two a priori density matrices  
$\sigma_0$ (called ``null hypothesis''), or $\sigma_1$ (called ``alternative hypothesis'') the goal is to decide by measurements on the outputs, which preparation is present. 
We prove ``Quantum  Stein's lemma'', 
which quantifies the behaviour of the error of optimal tests for this task in a situation, where large numbers of outputs of the systems are available for 
performing tests. \\
This lecture is central to the course for two reasons. A first one is, that hypothesis tests also make up for good  message transmission codes. This fact is already known
from classical Shannon information theory. However, this relation seems to be even more important for quantum systems, as we will see in subsequent sections. A second reason
is, that Stein's lemma allows a very simple and illuminating proof of the \emph{monotonicity of the quantum relative entropy under completely positive and trace preserving maps}, 
which is notoriously hard to prove otherwise. After all, this will be our entrance to several highly notrivial quantum entropic inequalities which are essential for proving 
major results in quantum Shannon theory. \\
The mentioned strategy to prove entropy inequalities starting from Quantum Stein's lemma is strongly inspired by the paper \cite{bjelakovic12a}, where also the relatively elementary proof 
of the result given below can be found. The interested reader also should consult that work. 
To formally settle the above described situation, assume, we are confronted with a preparation device which emits quantum systems pairwise uncorellated and 
additionally all being prepared according to the same density matrix. If this density matrix is $\gamma$, the mentioned properties of the preparation device ensure us, that 
the statistical behaviour of the joint quantum state of $n$ systems (``\emph{blocklength} $n$'') prepared is described by the density matrix 
\begin{align*}
\gamma^{\otimes n} := \underbrace{\gamma \otimes \cdots \otimes \gamma}_{n \ \text{times}}
\end{align*}
To settle the (asymetric) Hypothesis test problem, assume now, that the state $\gamma$ is unknown to the receiver of the systems. The receiver is provided with two 
a priori hypotheses in form of density matrices $\rho$ (``null hypothesis'') and $\sigma$ (``alternative hypothesis''), and tries by measurement on the outputs,
to decide, which of these hypothesis to accept. For a given test $\{E_0,E_1\}$ on the $n$-fold output system, two kinds of errors can happen 
\begin{enumerate}
	\item \textbf{First kind error:} The actual density matrix is $\rho$, but $\sigma$ is detected. This happens with probability 
	\begin{align*}
	\tr E_1 \rho^{\otimes n} = \tr (\bbmeins - E_0) \rho^{\otimes n}
	\end{align*}
	\item \textbf{Second kind error:} The density matrix is $\sigma$, but $\rho$ is detected. This happens with probability 
	\begin{align*}
	\tr E_0 \sigma^{\otimes n}
	\end{align*}
\end{enumerate}
\section{Quantum Stein's Lemma}
A common goal now is, to determine the optimal asymptotical behaviour of the second kind error for tests whose first kind error is below a threshold $\epsilon \in (0,1)$. 
For this reason, we define for each $\epsilon \in [0,1]$
\begin{align*}
 \beta_{\epsilon,n}(\rho, \sigma) := \inf\left\{\tr(a \sigma^{\otimes n}): 
 \ 0 \leq a \leq \bbmeins_\cH^{\otimes n} \ \text{and} \ \tr(a \rho^{\otimes n}) \geq 1 - \epsilon \right\}.
\end{align*}
To formulate the quantum version of Stein's Lemma, we need the following definition
\begin{definition}[Quantum relative entropy] \label{def:q_rel_ent}
 The \emph{quantum relative entropy} of a pair $(\rho,\sigma) \in \cS(\cK) \times \cS(\cK)$ is defined
 \begin{align*}
  D(\rho||\sigma) := \ \begin{cases}
                        \tr(\rho(\log \rho - \log \sigma)) & \text{if} \ \ker{\sigma} \subset \ker{\rho} \\
                        + \infty & \text{otherwise}
                       \end{cases}
 \end{align*}
\end{definition}
The following Theorem is the quantum theoretic generalization to Stein's Lemma.
\begin{theorem}[Quantum Stein's Lemma \index{Lemma!Quantum Stein's}] \label{thm:q_stein_lemma}
Let $\rho, \sigma \in \cS(\cH)$ density matrices with $\ker \sigma \subset \ker \rho$. For each $\epsilon \in (0,1)$, it holds
\begin{align*}
 \underset{n \rightarrow \infty}{\lim} \frac{1}{n} \log \beta_{\epsilon,n}(\rho, \sigma) \ = \ -D(\rho||\sigma)
\end{align*}
\end{theorem}
We will prove the above theorem in two portions. First, we prove the claim
\begin{align}
 \underset{n \rightarrow \infty}{\limsup} \frac{1}{n} \log \beta_{\epsilon,n}(\rho, \sigma) \ \leq \ -D(\rho||\sigma)  \label{thm:q_stein_lemma_achiev}
\end{align}
which implies, together with 
\begin{align}
 \underset{n \rightarrow \infty}{\liminf} \frac{1}{n} \log \beta_{\epsilon,n}(\rho, \sigma) \ \geq \ -D(\rho||\sigma) \label{thm:q_stein_lemma_converse}
\end{align}
the assertion of the theorem. 
\begin{subsection}{Types}
In this paragraph, we introduce a certain instance of "typical projections". 	
To proceed, we need the following definition
\begin{definition}[von Neumann entropy\index{entropy!von Neumann}] \label{def:von_neumann_entropy}
 The \emph{von Neumann entropy} of a density matrix $\rho \in \cS(\cH)$ is defined by
 \begin{align*}
  S(\rho) := - \tr(\rho \log \rho).
 \end{align*}
\end{definition}
\begin{remark}
Note, that $\log A$ is not defined, if is not of full rank. The above definition is to be understood using the convention $0 \log 0 = 0$.
\end{remark}
First, we introduce some notions, we will use. Let $\tau_1, \ \tau_2$ be density matrices on $\cH$, such that $\ker \tau_2 \subset \ker \tau_1$. We set
\begin{align*}
 M(\tau_1\| \tau_2) := - \tr \tau_1 \log \tau_2,
\end{align*}
which allows us to write
\begin{align*}
 D(\tau_1||\tau_2) 
 = \tr(\tau_1 \log \tau_1) - \tr(\tau_1 \log \tau_2)
 = -S(\tau_1) + M(\tau_1\|\tau_2), 
\end{align*}
and
\begin{align*}
 M(\tau_1||\tau_1) = S(\tau_1)
\end{align*}
in case that $\tau_1 = \tau_2$ holds. Set $d := \dim \cH$ and
\begin{align*}
 \tau_2 = \sum_{x=1}^d \mu(x) \ket{\phi_x}\bra{\phi_x}
\end{align*}
a spectral decomposition of $\tau_2$. For each $n \in \bbmN$, we obtain
\begin{align}
 \tau_2^{\otimes n} 
 & = \left(\sum_{x =1}^d \mu(x) \ket{\phi_x} \bra{\phi_x} \right)^{\otimes n} \nonumber 
 \nonumber \\
 & = \bigotimes_{i=1}^n \left(\sum_{x_1 =1}^d \mu(x_i) \ket{\phi_{x_i}} \bra{\phi_{x_i}}  \right) \nonumber  \nonumber \\ 
 & = \sum_{x_1=1}^d \cdots \sum_{x_n = 1}^d \mu(x_1) \cdots \mu(x_n) \ket{\phi_{x_1}} \bra{\phi_{x_1}} \otimes \cdots \otimes \ket{\phi_{x_n}} \bra{\phi_{x_n}} \nonumber \\
 & = \sum_{x^n \in [d]^n} \mu^n(x^n) \ket{\phi_{x^n}}\bra{\phi_{x^n}}.  \label{tens_prod_spect_dec}
\end{align}
The last equality is from introducing the notation
\begin{align*}
 \mu^n(x^n) : = \mu(x_1) \cdots \mu(x_n) = \prod_{i=1}^n \mu(x_i), \hspace{.3cm} \text{and} \hspace{.3cm} \phi_{x^n} := \phi_{x_1} \otimes \cdots \otimes \phi_{x_n}
\end{align*}
for each $x^n = (x_1, \dots, x_n) \in [d]^n$. In fact, $\{\phi_{x^n}: \ x^n \in [d]^n\}$ is an orthonormal basis in $\cH^{\otimes n}$. 
The right hand side of Eq. (\ref{tens_prod_spect_dec}) is a spectral decomposition of $\tau_2^{\otimes n}$. We define for each $\delta > 0$, $n \in \bbmN$ the set
\begin{align}
 T_{\delta,n}(\tau_1, \tau_2) 
 %:&= \left\{x^n \in [d]^n: \ M(\tau_1||\tau_2) - \delta < - \frac{1}{n} \log \mu^n(x^n) < M(\tau_1||\tau_2) + \delta \right\} \\
  &:= \left\{x^n \in [d]^n: \ 2^{-n(M(\tau_1||\tau_2) + \delta)} <  \mu^n(x^n) <  2^{-(M(\tau_1||\tau_2) - \delta)} \right\} \label{def:rel_typical_set}
 \end{align}
According to the set $T_{\delta,n}(\tau_1,\tau_2)$, we define the projector
\begin{align*}
 p_{\delta, n}(\tau_1,\tau_2) := \sum_{x^n \in T_{\delta,n}(\tau_1,\tau_2)} \ket{\phi_{x^n}} \bra{\phi_{x^n}}
\end{align*}
The next lemma collects some useful properties of the above type of projector. 
\begin{lemma}\label{lemma:steins_lemma_projection}
 Let $\tau_1, \tau_2 \in \cS(\cH)$, $\ker \tau_2 \subset \ker{\tau_1}$. For each $\delta > 0$, it holds with the abbreviation $p_n := p_{\delta,n}(\tau_1,\tau_2)$ 
 \begin{enumerate}
  \item $p_n \tau_2^{\otimes n} \ = \ \tau_2^{\otimes n}$ for all $n \in \bbmN$, \label{lemma:steins_lemma_projection_1}
  \item $p_n \tau_2^{\otimes n} p_n \ \leq \ 2^{-n(M(\tau_1||\tau_2)- \delta)} p_n$, \label{lemma:steins_lemma_projection_2}
  \item $p_n \tau_2^{\otimes n} p_n \ \geq \ 2^{-n(M(\tau_1||\tau_2)+ \delta)} p_n$, \label{lemma:steins_lemma_projection_3}
  \item $\lim_{n \rightarrow \infty} \tr p_n \tau_1^{\otimes n} \ = \ 1$. \label{lemma:steins_lemma_projection_4}
 \end{enumerate}
 \end{lemma}

 \begin{proof}
  The first three claims are obvious from the definitions. We will now show, that the fourth claim follows from an application of the law of large numbers. Define an 
  i.i.d. sequence 
  $
   X_1,\dots,X_n
  $
  of random variables with values in $[d]$ and probabilities
  \begin{align*}
   \Pr\left(X_i = x\right) = \braket{\phi_x, \tau_1 \phi_x}
  \end{align*}
  for each $i \in [n]$. We consequently have
  \begin{align*}
   \Pr\left(X_1 = x_1 \wedge \cdots \wedge X_n = x_n \right)
    \ &= \ \prod_{i=1}^n \Pr(X_i = x_i) \\
    \ &= \ \prod_{i=1}^n \braket{\phi_{x_i}, \tau_1 \phi_{x_i}} \\
    \ &= \ \braket{\phi_{x^n}, \tau_1^{\otimes n} \phi_{x^n}} 
   \end{align*}
   for each $x^n \in [d]^n$. Using the function $f: [d] \rightarrow \bbmR$ defined by 
   \begin{align*}
    f(x) := - \log \mu(x) &&(x \in [d]),
   \end{align*}
   we obtain another i.i.d. sequence $U_1,\dots,U_n$ of random variables, each defined by
   \begin{align*}
    U_i := f \circ X_i.   &&(i \in [n]).
   \end{align*}
   It holds
   \begin{align*}
    \bbmE U_i 
    \ &= \ \sum_{x \in [d]} \Pr(X_i = x) f(x) \\
    \ &= \ - \sum_{x \in [d]} \Pr(X_i = x) \log \mu (x) \\
    \ &= \ - \sum_{x \in [d]} \Pr(X_i = x) \braket{\phi_x, \log \tau_2 \phi_x} \\
    \ &= \ - \sum_{x \in [d]} \braket{\phi_x, \tau_1 \phi_x} \braket{\phi_x, \log \tau_2 \phi_x} \\
    \ &= \ - \tr(\tau_1 \log \tau_2)  \\
    \ &= \ M(\tau_1||\tau_2).
   \end{align*}
   Thus, it holds
   \begin{align*}
    \tr(\tau_1^{\otimes n} p_n) 
    &= \sum_{x^n \in T_{\delta,n}(\tau_1,\tau_2)} \tr(\tau_1^{\otimes n} \ket{\phi_{x^n}}\bra{\phi_{x^n}} \\
    &= \Pr\left((X_1,\dots,X_n) \in T_{\delta,n}(\tau_1,\tau_2)\right) \\
    &= \Pr \left(\left|\frac{1}{n} \sum_{i=1}^n U_i - M(\tau_1||\tau_2) \right| < \delta \right) \\
    &= \Pr \left(\left|\frac{1}{n} \sum_{i=1}^n U_i - n\bbmE(U_1) \right| < \delta \right)
   \end{align*}
   Taking the limit $n \rightarrow \infty$ in the chain of eqalities above, we obtain the fourth claim of the lemma by using the law of large numbers. 
 \end{proof}
  \end{subsection}
  \begin{subsection}{Proof - Achievability}
  To prove achievability in Quantum Stein's lemma, i.e. the inequality in Eq. (\ref{thm:q_stein_lemma_achiev}), we will define suitable tests which are essentially products of the projections $p_{n,\delta}(\rho, \sigma)$ 
  and $p_{n,\delta}(\rho,\rho)$ for some fixed $\delta$. These projections do not necessarily commute, therefore, we will need the following lemma. 
  \begin{lemma} \label{lemma:quantum_stein_matrix_ineq_1}
  Let $p,q \in \cL(\cK)$, $0 \leq p,q \leq \bbmeins_\cK$. It holds
  \begin{align*}
   \tr(\tau pqp) \ \geq \ \tr(\tau q) - 2 \sqrt{\tr \tau (\bbmeins_\cK - p)} .
  \end{align*}
  \end{lemma}
  \begin{proof}
   We abbreviate $\bbmeins_\cK$ with $\bbmeins$. It holds 
   \begin{align*}
    0 \ \leq \ (\bbmeins -p) q (\bbmeins - p) \ = \ -q + q(\bbmeins -p) + (\bbmeins - p)q + pqp,
   \end{align*}
   which is equivalent to 
   \begin{align*}
    q \ \leq \ q(\bbmeins -p) + (\bbmeins - p) q + pqp
   \end{align*}
   which in turn, by multiplying with $\tau$ and using the conjugation rule Lemma \ref{lemma:conjugation_rule} can be transformed to 
   \begin{align*}
    \tau^{\tfrac{1}{2}} q \tau^{\tfrac{1}{2}} \ \leq \ \tau^{\tfrac{1}{2}} q(\bbmeins -p) \tau^{\tfrac{1}{2}} + \tau^{\tfrac{1}{2}} (\bbmeins - p)q \tau^{\tfrac{1}{2}} + \tau^{\tfrac{1}{2}} pqp \tau^{\tfrac{1}{2}}.
   \end{align*}
   Note that
   \begin{align*}
    \tr \tau^{\tfrac{1}{2}} q (\bbmeins - p) \tau^{\tfrac{1}{2}} + \tr \tau^{\tfrac{1}{2}} (\bbmeins - p)q \tau^{\tfrac{1}{2}} \ 
     &= \ \tr \tau^{\tfrac{1}{2}} q (\bbmeins - p) \tau^{\tfrac{1}{2}} + \overline{\tr \tau^{\tfrac{1}{2}} (\bbmeins - p)q \tau^{\tfrac{1}{2}}} \\
     &= 2 \cdot Re \tr \tau^{\tfrac{1}{2}} q (\bbmeins - p) \tau^{\tfrac{1}{2}} \\
     &\leq 2 \cdot | \tr \tau^{\tfrac{1}{2}} q (\bbmeins - p) \tau^{\tfrac{1}{2}}| \\
     &\leq 2 \sqrt{\tr \tau q^2} \cdot \sqrt{\tr \tau (1-p)^2 }.
    \end{align*}
   Using monotonicity of the trace, we arrive at
   \begin{align*}
    \tr \tau q  
    & \leq \ \tr \tau pqp + \tr \tau^{\tfrac{1}{2}} q (\bbmeins - p) \tau^{\tfrac{1}{2}} + \tr \tau^{\tfrac{1}{2}} (\bbmeins - p)q \tau^{\tfrac{1}{2}} \\
    & \leq \ \tr \tau pqp + 2 \sqrt{\tr \tau q^2} \cdot \sqrt{\tr \tau (1-p)^2 }.\\
   \end{align*}
   Consequently, we have
   \begin{align*}
    \tr \tau pqp \ \geq \ \tr \tau q - 2 \sqrt{\tr \tau (\bbmeins - p)}, 
   \end{align*}
  as we desired to prove.
  \end{proof}
  
  \begin{proof}[Proof of achievability in Theorem \ref{thm:q_stein_lemma}, i.e. Eq. (\ref{thm:q_stein_lemma_achiev})]
  We show, for $\rho, \sigma \in \cS(\cH)$, $\ker \sigma \subset \ker \rho$, $\epsilon > 0$ 
  \begin{align*}
   \underset{n \rightarrow \infty}{\limsup} \frac{1}{n} \log \beta_{\epsilon, n}(\rho, \sigma) \ \leq \ - D(\rho||\sigma) 
  \end{align*}
  holds Fix numbers $\delta > 0$, $n \in \bbmN$. We define according to Lemma \ref{lemma:steins_lemma_projection} 
  %\begin{align*}
  $p_{n} := p_{\delta,n}(\rho, \rho)$,
  %\end{align*}
  $q_n := p_{\delta,n}(\rho, \sigma)$ and an effect
  %\begin{align*}
   $a_n := q_np_nq_n(\rho, \sigma)$,
  %\end{align*}
  which will serve as the hypothesis test for blocklength $n$. It holds
  \begin{align} 
   \tr a_n \rho^{\otimes n} \ = \ \tr q_np_nq_n\rho^{\otimes n} 
   \ \geq \ \tr p_n \rho^{\otimes n} - 2 \cdot \left(\tr (\bbmeins_{\cH}^{\otimes n} - q_n)\rho^{\otimes n}\right)^\frac{1}{2},  \label{quantum_stein_achiev_1}
  \end{align}
  where Lemma \ref{lemma:quantum_stein_matrix_ineq_1} was used. Each of the summands in Eq. (\ref{quantum_stein_achiev_1}) can be upper bounded by using properties of the projections stated in Lemma 
  \ref{lemma:steins_lemma_projection} with $\tau_1 = \tau_2 = \rho$. We have by Lemma \ref{lemma:steins_lemma_projection}.\ref{lemma:steins_lemma_projection_4} applied with $\tau_1 = \tau_2 = \rho$	
  \begin{align}
   \underset{n \rightarrow \infty}{\lim} \tr p_n \rho^{\otimes n} = 1, \label{quantum_stein_achiev_1_a}
  \end{align}
  and with $\tau_1 = \rho$, $\tau_2 = \sigma$
  \begin{align}
    \underset{n \rightarrow \infty}{\lim} \tr p_n \rho^{\otimes n} = 1. \label{quantum_stein_achiev_1_b}
  \end{align}
  Combination of (\ref{quantum_stein_achiev_1_a}) and (\ref{quantum_stein_achiev_1_b}) with (\ref{quantum_stein_achiev_1}) leads us to
  %\begin{align*}    
      $\underset{n \rightarrow \infty}{\lim} \ \tr a_n \rho^{\otimes n}  \ = \  1$. 
  %\end{align*}
   For each large enough $n$, we therefore have
   \begin{align}
    \tr(a_n \rho^{\otimes n}) \geq 1 - \epsilon \label{quantum_stein_achiev_2},
   \end{align}
   i.e. $a_n$ belongs to the feasible set of optimization for $\beta_{\epsilon,n}(\rho, \sigma)$. We will show, that the second kind error of $a_n$ is bounded in a suitable way for large enough blocklengths. Note, that 
   \begin{align}
	    \tr p_n q_n p_n 
	    \ \leq \ \tr p_n 
	    \ \leq \ 2^{-n(S(\rho) - \delta)} \cdot \tr p_n \rho^{\otimes n}. 
   \end{align}
   Consequently, it holds
   \begin{align*}
    \tr a_n \sigma^{\otimes n} \ 
    &= \ \tr q_n \sigma^{\otimes n} q_n p_n \\
	&\leq 2^{-n(M(\rho||\sigma) - \delta)} \cdot \tr p_n q_n p_n \\
	&\leq 2^{-n(D(\rho||\sigma)- 2\delta)}.
   \end{align*}
   Finally, we conclude, that for large enough $n$, 
   \begin{align*}
    \beta_{n, \epsilon}(\rho,\sigma)  \ \leq \ tr a_n \sigma^{\otimes n} \leq \exp(-n(D(\rho||\sigma) -2 \delta)),
   \end{align*}
   i.e. 
   \begin{align*}
    \underset{n \rightarrow \infty}{\limsup} \frac{1}{n} \log \beta_{\epsilon, n}(\rho, \sigma) \ \leq \ - D(\rho|| \sigma) + 2 \delta.
   \end{align*}
   Since $\delta$ was an arbitrary positive number, we are done. 
  \end{proof}
  \end{subsection}
  \begin{subsection}{Proof - Converse}
  For proving the converse statement (\ref{thm:q_stein_lemma_converse}), we will employ the following inequality, which can be derived from the one given in Lemma \ref{lemma:quantum_stein_matrix_ineq_1}.
  \begin{lemma}\label{lemma:quantum_stein_matrix_ineq_2}
   Let $p,q,u \in \cL(\cK)$, $0 \leq p,q \leq \bbmeins_\cK$, $\tau \in \cS(\cK)$, and $u$ a projection with $u\tau = \tau u$, and $\tau u \leq c u$ for some $c \in \bbmR_+$. Then 
   \begin{align*}
    \tr(pqp) \ \geq \ \frac{1}{c} \left( \tr \tau q  - 2 (\tr \tau(\bbmeins_\cK - p))^{\tfrac{1}{2}} - \tr \tau (\bbmeins_\cK - u)\right).
   \end{align*}
  \end{lemma}
  \begin{proof}
   It holds
   \begin{align*}
    \tr pqp \ 
    &= \ \tr upqp + tr (\bbmeins - u) pqp  \\
    &\geq \tr upqp \\ 
    &\geq \frac{1}{c} \ \tr \tau upqp \\
    & = \frac{1}{c} \left(\tr \tau pqp  - \tr \tau (\bbmeins -u) pqp \right) \\
    & \geq \frac{1}{c} \left(\tr \tau pqp  - \tr \tau (\bbmeins -u) pqp \right).
   \end{align*}
    Applying Lemma \ref{lemma:quantum_stein_matrix_ineq_1} to lower-bound $\tr \tau pqp$, we obtain the desired inequality.
  \end{proof}

  \begin{proof}[Proof of the converse to Quantum Stein's lemma]
  We fix $\epsilon \in (0,1)$ and show the inequality 
  \begin{align*}
   \underset{n \rightarrow}{\liminf} \frac{1}{n} \log \beta_{\epsilon, n}(\rho,\sigma) \ \geq \ -D(\rho||\sigma).
  \end{align*}
   Fix $\gamma > 1$. Let for $n \in \bbmN$, $a_n \in [0, \bbmeins_\cH^{\otimes n}]$ be any test, such that 
   \begin{align*}
    \tr q_n \rho^{\otimes n} \ & \geq  1 - \epsilon, \hspace{.2cm} \text{and}  \\
    \tr q_n \sigma^{\otimes n} \ & \leq \ \gamma \cdot \beta_{\epsilon,n}
   \end{align*}
   are simultaneously fulfilled (note, that such a text always exists). It holds (with $p_n, q_n$ as defined in the achievability proof)
   \begin{align*}
    \sigma^{\otimes n} \ 
    \geq \ q_n \sigma^{\otimes n} q_n
    \geq \ \exp\left(-n (M(\rho||\sigma) + \delta) \right) q_{n},
   \end{align*}
   by Lemma \ref{lemma:steins_lemma_projection_3} applied with $\tau_1 = \rho$, and $\tau_2 = \sigma$. Consequently,
   \begin{align}
    a_n^{\frac{1}{2}} \sigma^{\otimes n} a_n^{\frac{1}{2}} \ \geq \ \exp\left(-n ( M(\rho||\sigma) + \delta \right) a_n^{\frac{1}{2}} q_n a_n^{\frac{1}{2}} \label{quantum_stein_converse_3}
   \end{align}
   does hold, where the first inequality is by the conjugation rule (Lemma \ref{lemma:conjugation_rule}). Taking traces on both sides of the inequality in (\ref{quantum_stein_converse_3}), we arrive at
  \begin{align*}
    \beta_{\epsilon, n}(\rho, \sigma) \ \geq \ \gamma^{-1} \tr a_n \sigma^{\otimes n} \ \geq \  \gamma^{-1} \cdot \exp\left(-n (M(\rho||\sigma) + \delta \right) \cdot \tr a_n q_n
   \end{align*}
  Using Lemma \ref{lemma:steins_lemma_projection} with $\tau_1 = \tau_2 = \rho$ leads us to 
  \begin{align}
   p_n \rho^{\otimes n} p_n \ \geq \ 2^{-n (S(\rho) - \delta)} p_n. \label{quantum_stein_converse_4}
  \end{align}
  With the inequality in (\ref{quantum_stein_converse_4}), the conditions of Lemma \ref{lemma:quantum_stein_matrix_ineq_2} are fulfilled  with the assignments $u = p = q_n$, $q = a_n$. $\tau = \rho$, 
  and $c = \exp(-n(S(\rho) - \delta))$. It holds
  \begin{align*}
   \tr q_n a_n q_n \ &\geq \ \exp(n(S(\rho) - \delta))  \\
			    & \cdot \left(\tr \rho^{\otimes n} a_n  - 2 \left(\tr \rho^{\otimes n}(\bbmeins_{\cH}^{\otimes n} - q_n))\right)^{\tfrac{1}{2}} - \tr \rho^{\otimes n} 
   (\bbmeins_\cH^{\otimes n} - q_{n})\right).
  \end{align*}
  Note, that the first term in brackets on the r.h.s. of the above inequality approaches one, while the others go to zero. We conclude, that for large enough $n$, we have
  \begin{align*}
  \tr q_n a_n q_n  \ &\geq \ \frac{1- \epsilon}{2} \cdot \exp(n(S(\rho) - \delta)).
  \end{align*}
  Putting everything together and taking logarithms, we arrive at 
  \begin{align*}
   \frac{1}{n} \log \beta_{\epsilon, n}(\rho, \sigma) \ \geq    \frac{1}{n} \log\frac{1- \epsilon}{2\gamma} \cdot (-D(\rho||\sigma) - 2 \delta)
  \end{align*}
  Taking the limes inferior on both sides of the above inequality gives the desired result.
  \end{proof} 
  \end{subsection}
  \section{Exercises}
  
  \begin{exercise}
   Use the strategy of the above proof to Stein's Lemma to provide yourself with a proof of the classical version of Stein's lemma. Hint: In the classical case, the Lemmas \ref{lemma:quantum_stein_matrix_ineq_1} and 
   Lemma \ref{lemma:quantum_stein_matrix_ineq_2} can be replaced by usage of the union
    bound.
  \end{exercise}
  
%  \begin{exercise}[Singular case of Quantum Stein's Lemma]
%   sdfs
%  \end{exercise}








