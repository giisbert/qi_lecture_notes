These notes where written as an accompanying document for a quantum information theory master's course held at the ET-IT department at the Munich Technical University, and therefore adresses primarly an audience with classical information theory backround rather than a formal education in quantum mechanics.\\
These notes (as well as the course) intend to provide access to the topic for the uninitiated. Therefore, the focus lies on careful introduction of the underlying structure avoiding to praise the folklore. A downside of this approach is a loss in pace. The contents of the course in consequence only cover a very small fraction of what is nowadays canonical. The motivated reader is referred to the exercises.\\
Especially do these notes not intend to be a textbook on the topic nor should it be regarded as a replacement. The reader is pointed to consult one of the excellent textbooks on the field (we here just mention as examples the books of  Wilde \cite{wilde13}, and Holevo \cite{holevo12a}.) 
Next, we give some overview of the topics of the lectures.  \vspace{3ex} \\
\textbf{Lecture 1.} We introduce the basic entities of quantum theory (for finite-degree systems), and comfort ourselves with its structure by recognizing it as a statistical theory.  \\
\textbf{Lecture 2.} Quantum systems with more than one subsystems are considered. We introduce direct sum and tensor product spaces to describe them. Mathematical tools for describing such systems are introduced. We encounter the first severe "nonclassical" features: Purifications, Entanglement, Noncompatibility. \\
\textbf{Lecture 3.} As a first asymptotical quantum statistical Problem, we discuss the Quantum Hypotheses Testing Problem. We prove a quantum version of Stein's Lemma which acquaints us with the quantum relative entropy.  \\
\textbf{Lecture 4.} Quantum channels, the broades class of allowed transformations of quantum states are introduced as completely positive and trace preserving linear maps. 
We add two important equivalent representation theorems (Kraus decomposition and Stinespring dilation) to our toolbox.  \\
\textbf{Lecture 5.} Source compression of discrete memoryless quantum sources ins considered. The quantum source compression theorem is proved. We get to know the quantum Fidelity, and entanglement fidelity as meaningful figures of merit. The optimal compression rate is determined in terms of the von Neumann entropy. \\
\textbf{Lecture 6.} Transmission of classical messages over disrete and memoryless classical quantum and quantum channels is discussed. The Holevo Theorem is proven to 
determine the classical message transmission capacity. The coding theorem as well as the converse theorem will be derived as implications of Quantum Stein's Lemma. \vspace{3ex}

While it seems, that the quantum Hypotheses testing problem and Quantum Stein's lemma are usually not regarded as part of the canonical arsenal of first-course quantum information theory topics. In this course, Quantum Stein's lemma plays a central role. On one hand, good, capacity achieving classical message transmission codes for quantum channels can be derived from certain good hypothesis tests. On the other hand, Quantum Stein's lemma allows for a elementary "information-theoretic" proof of the monotonicity of quantum relative entropy under quantum channels - which helps to derive most of the important entropy inequalities in quantum information theory. \newline 
The approach to teaching quantum information theory (and teaching it at all) is inspired by I. Bjelakovi\'c and R. Siegmund-Schultze's paper \cite{bjelakovic12a}, which is strongly recommended as additional reference.













