In the preceding lecture, we introduced the basic quantum-theoretic entities for given quantum system with Hilbert space $\cH$. In this lecture, we consider quantum systems
which are composed of two or more \emph{subsystems}. E.g. we need to be able to describe the statistics of experiments where two electron spins (each with Hilbert space $\cH = \bbmC^2$) are involved. 
It is one of the major tasks in information theory, to quantify correlations between such subsystems and optimally exploit statistical properties of composite systems for usage in communication protocols. \\
Remembering classical statistical theory (with finite sample spaces), the state of a system composed of $n < \infty$ subsystems with alphabets $\cX_1,\dots,\cX_n$ the set of states on that system 
is the set $\cP(\cX_1 \times \cdots \times \cX_n)$ of probability distributions on the cartesian product of $\cX_1,\dots,\cX_n$. We notice, that the pure states are exactly the products of point measures, $\{\delta_{x^n}: \ x^n \in \cX^n \}$ with 
\begin{align*}
 \delta_{x^n}(y^n) := \prod_{i=1}^n \delta_{x_i}(y_i) && (x^n = (x_1,\dots,x_n), y^n = (y_1,\dots,y_n)).
\end{align*} 
Taking as an example $n=2$, we can define to each $p \in \cP(\cX \times \cX_2)$ the \emph{marginal states} $p_1 \in \cP(\cX_1), p_2 \in \cP(\cX_2)$ by
\begin{align*}
p_1(x_1) := \sum_{x_2 \in \cX_2} \ p(x_1,x_2), \hspace{.3cm} \text{and} \hspace{.7cm} p_2(x_2) := \sum_{x_1 \in \cX_1} p(x_1,x_2)  
\end{align*}
which recover the statistical properties of the individual subsystems. The systems are statistically \emph{independent}, if the state $p$ is a product distribution, i.e. with some $p_1 \in \cP(\cX_1), \ p_2 \in \cP(\cX_2)$ it holds
\begin{align*}
   p(A \times B) \ = \ p_1(A)\cdot p_2(B)
\end{align*}
for all $A \subset \cX_1$, $B \subset \cX_2$ (the usual notation is then "$p = p_1 \otimes p_2$".) We notice, that each $w \in \cP(\cX_1 \times \cX_2)$ can be written as a convex combination of product states via
\begin{align*}
 w \ = \ \sum_{x^n \in \cX_1 \times \cdots \times \cX_n} w(x^n)  \ \delta_{x^n}.
\end{align*}
Regarding a composite quantum system with subsystem Hilbert spaces $\cH_1,\dots, \cH_n$, we need to define the states on that Hilbert space. Therefore, the cartesian product $\cH_1 \times \dots \times \cH_N$ has to be equipped with a Hilbert space structure. 
\begin{subsection}{Mathematical Interlude -- Direct sums and tensor products}
 In this course, we will make extensive use of two different possibilites to provide the cartesian product 
 \begin{align*}
  V_1 \times \cdots \times V_N \ := \ \left\{ (v_1,\dots, v_N): \  v_i \in V_i, \ 1 \leq i \leq N \right\}
 \end{align*}
 of linear spaces $V_1,\dots, V_n$ with a linear structure, the \emph{direct sum}\index{linear space!direct sum} $V_1 \oplus \cdots \oplus V_n$ and the \emph{tensor product} $V_1 \otimes \cdots \otimes V_n$\index{linear space!tensor product}. Since we are always dealing with Euclidean
 spaces, we assume $V_i$ to be equipped with a scalar product $\braket{\cdot,\cdot}_i$, $i = 1,\dots, n$. 
 \end{subsection}
 \begin{subsection}{Direct sum spaces}
 The direct sum $V_1 \oplus \cdots \oplus V_N$ of $V_1, \dots, V_N$ is defined as the set
 \begin{align*}
   \left\{ \left(\begin{array}{c} v \\ w   \end{array}\right): \ v \in V_1,\ w \in V_2 \right\}
 \end{align*}
 together with the obvious linear structure inherited from the component spaces by using addition and scalar multiplication component-wise, i.e.
 \begin{align*}
  \lambda \cdot \left(\begin{array}{c} v_1 \\ \vdots \\ v_N   \end{array}\right) \ 
  & =  \ \left(\begin{array}{c} \lambda \cdot  v_1 \\ \vdots \\ \lambda \cdot v_N   \end{array}\right) &(\lambda \in \bbmC, v_1 \in V_i, \ 1 \leq i \leq N), \ \text{and} \\ 
  \left(\begin{array}{c} v_1 \\ \vdots \\ v_N   \end{array}\right) + \left(\begin{array}{c} w_1 \\ \vdots \\ w_N   \end{array}\right) \
  & = \ \left(\begin{array}{c} v_1 + w_1 \\ \vdots \\  v_N + w_N   \end{array}\right) &(v_i, w_i \in V_i, \ 1 \leq i \leq N).
 \end{align*}
 We may also define a scalar product $\braket{\cdot,\cdot}_{1 \oplus \dots \oplus N}: V_1 \cdots \oplus V_N \rightarrow \bbmC$ by setting
 \begin{align*}
  \left\langle \left(\begin{array}{c}v_1 \\ \vdots \\ v_N \end{array}\right), \left(\begin{array}{c}w_1 \\ \vdots \\ w_N \end{array} \right) \right\rangle_{1 \oplus \dots \oplus N} := \sum_{i=1}^N \braket{v_i,w_i}_i 
 \end{align*}
 for all $v_i,w_i \in V_i$, $1 \leq i \leq N$. On the other hand, if $\cH$ is a Hilbert space, and $V_1,\dots, V_N \subset \cH$ are pairwise orthogonal linear subspaces such that for each $h \in \cH$
 exist uniquely $h_1 \in V_1, \dots, \ h_N \in V_N$ such that $h = h_1 + \dots + h_N$, then $\cH$ is isomorphic to $V_1 \oplus \dots \oplus V_N$. We can define injections 
 \begin{align*}
  I_j \in \cL(V_j, \cH), \hspace{.2cm}  v \mapsto I_j (v) = (0, \dots,v, \dots, 0) &&(v \in V_j).
 \end{align*}
 Notice, that the adjoints are given by $I_j^\ast(h) = P_j h$, where $P_j$ is the projector onto $V_j$ for $j \in [N]$. Choosing an appropriate orthonormal basis, each $A \in \cL(\cH)$ can be 
 written as a \emph{block matrix} \index{matrix!block}
 \begin{align*}
  A = \left(\begin{array}{ccc}
  A_{11} & \cdots & A_{1N} \\ 
  \vdots & \ddots& \vdots \\ 
  A_{N1} & \cdots & A_{NN} \end{array}\right) 
 \end{align*}
 where $A_{ij} = I_j^\ast AI_i\in \cL(V_i, V_j)$ is a smaller matrix for all $i,j \in [N]$.
 \begin{exercise}
  Show, that 
  if $\{v_i\}_{i=1}^{\dim V_1}\subset V_1$ and $\{w_j\}_{j=1}^{\dim V_2} \subset V_2$ are orthonormal bases in $V_1$, $V_2$, then
   \begin{align}
    \left\{\left(\begin{array}{c} v_i \\ 0 \end{array} \right) \right\}_{i=1}^{\dim V_1} \ \cup \ \left\{\left(\begin{array}{c} 0 \\ w_i \end{array} \right) \right\}_{j=1}^{\dim V_2}
   \end{align}
   is an orthonormal basis in $V_1 \oplus V_2$ according to the corresponding scalar product. In particular, $\dim V_1 \oplus V_2 = \dim V_1 + \dim V_2$.
 \end{exercise}
 \end{subsection}
 \begin{subsection}{Tensor product spaces}
  Another way to equip the cartesian product of $V_1, \dots, V_N$ with a linear structure is forming the tensor product. First, we note, that each 
  pair $(v_1,\dots, v_N) \in V_1 \times \cdots \times V_N$ defines a $N$-linear function $v_1 \otimes  \otimes v_N: V_1 \times \dots \times V_N \rightarrow \bbmC$ by
  \begin{align*}
   v_1 \otimes \dots \otimes v_N (x_1,\dots, x_N) := \braket{v_1, x_1}_1\cdot \dots \cdot \braket{v_N,x_N}_N. 
  \end{align*}
  The definitions can be obviously extended to form linear combinations of these \emph{elementary tensors}, i.e. the rules
  \begin{align*}
    \lambda (v \otimes w) 
    &= (\lambda v) \otimes w = v \otimes (\lambda w)  &(\lambda \in \bbmC, v \in V_1, w \in V_2) , \\
    (v_1 + v_2) \otimes w 
    &= v_1 \otimes w + v_2 \otimes w &(v_1,v_2 \in V_1, w \in V_2) \\
    v \otimes (w_1 + w_2)
    &= v \otimes w_1 + v \otimes w_2 &(v \in V_1, w_1, w_2 \in V_2) 
  \end{align*}
  apply (the rules above are formulated for $N=2$, but it is quite clear how the version for general finite $N$ looks). We can form formal linear combinations 
  \begin{align*}
   \sum_{i=1}^N \alpha_{i_1\dots i_N} v_{i_1} \otimes \cdots \otimes v_{i_N}
  \end{align*}
  of elementary tensor product vectors with coefficients $\alpha_{i_1\dots i_N} \in \bbmC$. The set 
  \begin{align*}
   V_1 \otimes \cdots \otimes V_N := \spann \{v_1 \otimes \cdots \otimes v_N: \  v_i \in V_i, \ 1 \leq i \leq N\}
  \end{align*}
  is called the tensor product of $V_1$ and $V_2$. We also can extend these structures to the linear maps. With spaces $V_1, \dots, V_N, \tilde{V}_1, \dots  
  \tilde{V}_N$, we define the tensor product $\cL(V_1,\tilde{V}_1) \otimes \dots \otimes \cL(V_N, \tilde{V}_N)$ accordingly. For each $A_1,\dots,A_N$, $A_i \in \cL(V_i,\tilde{V_i})$, $A_1 \otimes \cdots \otimes A_N$ is the map defined by 
  \begin{align*}
   (A \otimes \cdots \otimes A_N)(v_1 \otimes \cdots \otimes v_N) 
   = A_1v_1 \otimes \cdots \otimes A_N v_N
  \end{align*}
  for each $v_1 \in V_1, \dots, v_N \in V_N$. We have
  \begin{align*}
     \cL(V_1 \otimes V_2, \tilde{V}_1 \otimes \tilde{V}_2) 
   &= \ \cL(V_1, \tilde{V}_1) \otimes \cdots \otimes \cL(V_N, \tilde{V}_N)  \\
   &= \ \spann \{A_1 \otimes \cdots A_N: \ A_i \in \cL(V_i, \tilde{V}_i), \ 1 \leq i \leq N\}.
  \end{align*}
  
 \end{subsection}
  \begin{subsection}{Isomorphisms and representations} 
 
 In calculations and proofs, it is sometimes advantageous, to transfer the objects under investigation to another space. In the following we collect some isomorphisms
 which are often used. The definitions below are defined for members of the canonical euclidean orthonormal basis, but extend -- by linearity -- to all elements of the underlying spaces. 
 \begin{enumerate}
  \item 
 \begin{align*}
   \Sigma: \ \bigoplus_{i=1}^N \cH \   
   \rightarrow \ \cH \otimes \bbmC^N,  \hspace{.8cm}
    \left(\begin{array}{c} v_1 \\ \vdots \\ v_N \end{array} \right)  \mapsto \sum_{i=1}^N v_i \otimes e_i 
 \end{align*}
 \item 
 \begin{align}
  \Gamma: \ \cL(\cH_1,\cH_2) \ \rightarrow \ \cH_1 \otimes \cH_2 \hspace{.8cm} \ket{v}\bra{w} \  \mapsto \ \ket{\overline{w}} \otimes \ket{v} \label{def:iso_gamma}
 \end{align}
 \item 
 \begin{align*}
  \Lambda: \ \cL(\bbmC^M \otimes \bbmC^N) \ 
   \rightarrow  \ \cL\left(\bigoplus_{i=1}^M \bbmC^N\right) \ \hspace{.8cm}
   \ket{e_i} \bra{e_j} \otimes \ket{e_k}\bra{e_l} \
   \mapsto  I_i^\ast \ket{e_k}\bra{e_l} I_j
  \end{align*}
 \end{enumerate}
With the preparations, we are able to define quantum states on composite systems.
\begin{definition} 
 The set of states of a composite system with $N < \infty$ subsystems, $\cH_i$ being the Hilbert space of the $i$-th subsystem is given by the set
 \begin{align*}
  \cS\left(\bigotimes_{i=1}^N \ \cH_i \right)
 \end{align*}
 of density matrices on the tensor product $\bigotimes_{i=1}^{N} \ \cH_i := \cH_1 \otimes \dots \otimes \cH_N$ of the spaces $\cH_1, \dots \cH_N$. 
\end{definition}
A corresponding map to form ``marginal states'' on subsystems is given by the partial trace, which we define for notational simplicity for bipartite
systems (i.e. composite systems consisting of two subsystems). 
\begin{definition}[Partial trace] \index{partial trace} \index{trace!partial}
 For two Hilbert spaces $\cH_1, \cH_2$, the partial trace (over $\cH_2$) is the linear map
 \begin{align*}
  \tr_{\cH_2}: \ \cL(\cH_1 \otimes \cH_2) \rightarrow \cL(\cH_1) 
 \end{align*} 
 defined by 
 \begin{align*}
  \tr_{\cH_2}(\ket{u \otimes v}\bra{ w \otimes x}) := \braket{x,v} \ket{u}\bra{w} &&(u,w \in \cH_1, \ v,x \in \cH_2)
 \end{align*}
 plus linear extension. The corresponding partial trace over $\cH_1$, $\tr_{\cH_1}$ is defined analogeously. 
\end{definition}
 We demonstrate by example, how the partial trace can be calculated. Let $\cH_1, \cH_2$ be Hilbert spaces, $\dim \cH_i = d_i$, $i = 1,2$.  
 Let $A \in \cL(\cH_1 \otimes \cH_2)$, with orthonormal tensor basis decomposition
 \begin{align*}
  A = \sum_{i,j=1}^{d_1} \sum_{k,l=1}^{d_2} \ a_{ijkl} \ \ket{f_i \otimes g_k} \bra{f_j \otimes g_l}. 
 \end{align*}
 Then, we calculate
 \begin{align*}
  \tr_{\cH_2}(A) 
  &= \sum_{i,j=1}^{d_1} \sum_{k,l=1}^{d_2} \ a_{ijkl} \ \tr_{\cH_2}(\ket{f_i \otimes g_k} \bra{f_j \otimes g_l}) \\
  &= \sum_{i,j=1}^{d_1} \sum_{k}^{d_2} \ a_{ijkl} \ \braket{f_l, f_k} \cdot \ket{f_i} \bra{f_j} \\
  &= \sum_{i,j=1}^{d_1}  \left(\sum_{k}^{d_2} \ a_{ijkk}\right)  \cdot \ket{f_i} \bra{f_j}.
 \end{align*}
  Note, that the definition of partial traces easily extends to composite systems with more than two subsystems. Let $N$ parties (each of them with Hilbert space 
  $\cK_i$ assigned, share a system with Hilbert space $\cK_1 \otimes \dots \otimes \cK_N$. To apply the partial trace on the $j$-th system one uses the definition
  above with $\cH_1 = \bigotimes_{i \neq j} \cK_i$, $\cH_2 := \cH_i$ and so on...
  \begin{proposition}
  Let $A \in \cL(\cH_1 \otimes \cH_2)$. It holds 
  \begin{enumerate}
   \item $\tr(A) = \tr(\tr_{\cH_1}(A)) = \tr(\tr_{\cH_2}(A))$,
   \item $\tr(\tr_{\cH_2}(A)B) = \tr(A(B \otimes \bbmeins_{\cH_2}))$,
   \item $A \geq 0 \ \Rightarrow \ \tr_{\cH_2}(A) \geq 0$,
   \item $\rho \in \cS(\cH_1 \otimes \cH_2) \ \Rightarrow \ \tr_{\cH_2}(\rho) \in \cS(\cH_2)$. 
  \end{enumerate}
 \end{proposition}
  \begin{proof}
   The claims of the proposition can be verified by straightforward calculation, left as an exercise. As an example, we show the second claim. Let $x \in \cH_1$, $\|x\| =1$. Note, that 
   \begin{align*}
    P:= \ket{x}\bra{x} \otimes \bbmeins_{\cH_2}
   \end{align*}
   is a projection. We have
   \begin{align*}
    \braket{x, \tr_{\cH_2}(A) x}  
    \ =\ \tr(\ket{x}\bra{x} \tr_{\cH_2}(A)) 
    \ =\ \tr(A^{\tfrac{1}{2}}PA^{\tfrac{1}{2}}) 
    \ \geq\  0.
   \end{align*}
  \end{proof}
  Here, we may point out a significant difference between the concept of a bipartite state in classical theory and quantum theory. As noticed earlier, each classical bipartite state (i.e. probability distribution on a product alphabet) 
   can be written as a convex combination of product distributions. Quantum probability 
   offers an additional class of states beyond. 
  \begin{definition} \label{def:bipartite_state_classification}
   A state $\rho \in \cS(\cH_1 \otimes \cH_2)$ is called
   \begin{enumerate}
    \item[(i)] \emph{"uncorrelated", "product state"} \index{quantum state!product}, if it can be written $\rho = \rho_1 \otimes \rho_2$ for some $\rho_1 \in \cS(\cH_1)$, $\rho_2 \in \cS(\cH_2)$. 
    \item[(ii)] \emph{"separable state"} \index{quantum state!separable}, if it admits the form 
    \begin{align*}
     \rho = \sum_{i=1}^N \lambda_i \ \rho_1^{(i)} \otimes \rho_2^{(i)}
    \end{align*}
    with $N \in \bbmN$, $\lambda_i \in (0,1)$, $\rho_1^{(i)} \in \cS(\cH_1)$, and $\rho_2^{(i)} \in \cS(\cH_2)$ for all $i \in [N]$, $\sum_{i=1}^N \lambda_i = 1$. 
    \item[(iii)] \emph{entangled} \index{quantum state!entangled} otherwise.
   \end{enumerate}
  \end{definition}
   \begin{remark}
    One could ask about infinite or even uncountable convex combinations of product states. These are identified as uncorrelated by means of Caratheodory's theorem. 
    It asserts, that for given convex subset $A \subset \bbmR^d$, each element $x \in A$ can be written as a convex combination of at most $d+1$ extremal elements in 
    $A$. Consequently, each separable state can be written as a finite convex combination of not more than $2d+1$ product states. 
   \end{remark}
   A closer look at Definition \ref{def:bipartite_state_classification} reveals, that beyond the convex combinations of product states (of which the product states itself are a trivial
   subclass), there is another class of states having no classical counterpart. The class of entangled states is indeed nonempty, as the following example demonstrates.
	Let two unit vectors $\varphi_1, \varphi_2 \in \bbmC^d$ with $\varphi_1 \perp \varphi_2$, and nonzero coefficients $\alpha, \beta \in \bbmC$, $|\alpha|^2 + |\beta|^2 = 1$ be given. We
	show, that 
	\begin{align}
	\eta  \ = \ \alpha \varphi_1 \otimes \varphi_1 + \beta \varphi_2 \otimes \varphi_2 \label{ex_entangled_state_1}
	\end{align}
    is state vector of an entangled state. In fact, the assumption, that $\eta$ is separable leads to a contradiction. Let $\{\varphi_1,\varphi_2,\dots,\varphi_d\}$ be an extension of $\varphi_1,\varphi_2$ to an orthonormal basis in $\bbmC^d$. Since $\eta$ is pure and separable, it takes the form of a product vector
    \begin{align}
     \eta \ = \ \left(\sum_{i=1}^d c_i \varphi_i \right) \otimes \left(\sum_{i=1}^d \tilde{c}_j \varphi_j\right) \ = \ \sum_{i,j=1}^d c_i \tilde{c}_j \varphi_i \otimes \varphi_j.
     \label{ex_entangled_state_2}
    \end{align}
    Comparing coefficients in (\ref{ex_entangled_state_1}) and (\ref{ex_entangled_state_2}), shows, that 
    \begin{align}
     c_1 \cdot \tilde{c}_1 \ = \alpha, \hspace{.3cm} c_2 \cdot \tilde{c}_2 \ = \beta, \hspace{.3cm} c_i \cdot \tilde{c_j} \ = \ 0 \ \text{for} \  i \neq j \ \text{or} \ i,j > 2.
    \end{align}
    As a consequence of the above equalities, $\alpha \cdot \beta = c_1 \cdot \tilde{c}_2 \cdot \tilde{c}_1 \cdot c_2  = 0$, which is a contradiction to $\alpha, \beta \neq 0$. We record
   \begin{example}
	A pure bipartite state $\eta = \alpha \cdot \varphi_1 \otimes \varphi_1 + \beta \cdot \varphi_2 \otimes \varphi_2$, $|\alpha|^2 + |\beta|^2 = 1$ is entangled if and only if $\alpha, \beta \neq 0$.
   \end{example}

	\end{subsection}
	\begin{subsection}{Schmidt decomposition}
    Especially useful is the following ``polar'' representation of vectors on a tensor product. 
   \begin{theorem}\label{theorem:schmidt_decomposition} \index{decomposition!Schmidt -}
    Let $a \in \cH_1 \otimes \cH_2$. Then there exist ortonormal systems $\{v_i\}_{i=1}^m \subset \cH_1$, $\{w_i\}_{i=1}^m \subset \cH_2$ and numbers 
    $\alpha_1 \geq \alpha_2 \geq \dots \geq \alpha_m > 0$, such that 
    \begin{align*}
     a = \sum_{i=1}^m \sqrt{\alpha_i} v_i \otimes w_i \label{theorem:schmidt_decomposition_1}
    \end{align*}
    holds. 
   \end{theorem}
   \begin{remark}
    \begin{enumerate}
     \item $m$ is usually called the \emph{Schmidt number}\index{Schmidt number} of $a$, $\sqrt{\alpha_1}, \dots, \sqrt{\alpha_m}$ the \emph{Schmidt coefficients}
     \index{Schmidt coefficients}.
     \item It holds 
     \begin{align*}
      A_1 := \tr_{\cH_2}(\ket{a}\bra{a}) &= \sum_{i=1}^{m} \alpha_i \ket{v_i}\bra{v_i} \\
      A_2 := \tr_{\cH_1}(\ket{a}\bra{a}) &= \sum_{i=1}^{m} \alpha_i \ket{w_i}\bra{w_i}. 
     \end{align*}
      In particular the Schmidt coefficients are the nonzero eigenvalues (counted with their multiplicities), and the vectors in the orthonormal systems which appear in 
      the Schmidt decomposition are the corresponding eigenvectors.
    \end{enumerate}
	\end{remark}
    Though formulated for general bipartite vectors, we will use the Schmidt decomposition most of the time for pure quantum states. 
    The Schmidt decomposition Theorem is essentially a reformulation of the singular value decomposition, see Theorem \ref{thm:svd}. 
    \begin{proof}[Proof of Theorem \ref{theorem:schmidt_decomposition}]
	 Using the inverse of the linear isomorphism $\Gamma$ from (\ref{def:iso_gamma}), we
	 obtain a matrix $A = \Gamma^{-1}(\ket{a})$. Let
	 \begin{align*}
	 A = \sum_{i,j=1}^{m} \sqrt{\alpha_i} \ket{v_i}\overline{\bra{w_i}}
	 \end{align*}
	 be a singular value decomposition of $A$. Then 
	 \begin{align*}
	  a \ = \Gamma \circ \Gamma^{-1} a = \sum_{i=1}^m \sqrt{\alpha_i} \ \Gamma(\ket{w_i}\overline{\bra{v_i}}) = \sum_{i=1}^m \sqrt{\alpha_i} v_i \otimes w_i.
	 \end{align*}
    \end{proof}
	As we will see in forthcoming lectures, Schmidt decomposition is a very convenient way to write a pure bipartite state in proofs and calculations. Unfortunately, there seems to be no general satisfactory extension of Theorem \ref{theorem:schmidt_decomposition} to spaces with more than two tensor factors. Nevertheless, Schmidt decompositions allows us to construct "purifications" of states, we define.
	\begin{definition}[Purification] \index{purification}		
	  Let $\rho \in \cS(\cH)$ be a state, and $\cK$ be an additional Hilbert space. The pure state $\ket{\Psi}\bra{\Psi}$ is a \emph{purification} of $\rho$ if, $\tr_\cK \ket{\Psi}\bra{\Psi} = \rho$. 
	\end{definition}
	The Schmidt decomposition offers a nice principle to construct purifications of a state $\rho$. By the remark following Theorem \ref{theorem:schmidt_decomposition}, the Schmidt coefficients and the orthonormal vectors in one tensor factor in Schmidt decompositions are obtained from a spectral decomposition of $\rho$. The remaining Schmidt vectors can be chosen freely on any Hilbert space of sufficient dimension (see Exercise \ref{ex:schmidt_dec}). 
		\begin{observation} \label{obs:mon_ent}
		A quantum system in a pure state can only be uncorrelated to the ``outside world'', i.e. if $\rho \in \cS(\cH \otimes \cK)$ is such that $\tr_\cK \rho$ is pure, it is necessarily a product state.
	\end{observation}
	The above observation also holds in for the classical statistical theory. However, in quantum information theory, this insight becomes a powerful tool when combined with the possiblity to "purify" quantum systems. The bipartite pure state resulting from purification captures all correlations of the system "to the outside world". This is a "quantum feature", since purifying systems is not possible in classical theory (see Exercise \ref{ex:class_pur}.)
	\end{subsection}
     \begin{section}{Supplement: Entanglement witnesses}
      \begin{definition}[Entanglement witness]
      A matrix $A \in \cL^h(\cH_A \otimes \cH_B)$ is called an \emph{entanglement witness} \index{entanglement!witness}, if it is not positive semidefinite, and 
      \begin{align}
       \tr A(\rho \otimes \sigma) \geq 0 
      \end{align}
      for all $\rho \in \cS(\cH_A)$, $\sigma \in \cS(\cH_B)$. An entangled state $\tau$ is said to be \emph{detected} by $A$, if 
      \begin{align}
       \tr A \tau < 0.
      \end{align}
     \end{definition}

     \begin{definition}
      A hyperplane in $\bbmR^m$ is a set of the form
      \begin{align}
       H(\xi, k) := \{z \in \bbmR^m: \braket{\xi,z} = k\},
      \end{align}
      where $\xi \in \bbmR^m \setminus \{0\}$, and $k \in \bbmR$. $H(\xi,k)$ is said to separate two sets $S_1, S_2 \subset \bbmR^m$, if 
     \end{definition}
     Some properties 
     \begin{enumerate}
      \item $H(\xi,k) \perp span \{\xi\}$
      \item $\forall x,y \in H(\xi,k) ,\lambda \in \bbmR$, it holds $\braket{\lambda, \xi, x -y}$
     \end{enumerate}
     
     
     \begin{theorem}[Separating Hyperplanes] \label{thm:separating_hyperplanes}
     If $S$ is a closed convex subset of $\bbmR^m$, and $x_0 \notin S$, then there exists a hyperplane separating $x_0$ and $S$.
     \end{theorem}
  
     \begin{proof}
      First step. We show, that there is a unique $s_0 \in S$ minimizing the distance $|s-x_0|$. Such an $s_0$ exists, because $S$ is closed. Namely, choosing a sequence $\{s_i\}_{i \in \bbmN}$ such that 
      \begin{align}
       \underset{k \rightarrow \infty}{\lim}|s_k - x_0| = \underset{s \in S}{\inf}|s - x_0|
      \end{align}
      holds. By the Bolzano-Weierstrass Theorem (``Every bounded sequence has a convergent subsequence''), we find a subsequence $\{s_{k(j)}\}_{j=1}^\infty$, which converges to an $s_0 \in \bbmR^m$, which is 
      also a member of $S$, since by hypothesis, $S$ is closed. Consequently,
      \begin{align}
       |s_0 - x_0| \ = \ \underset{j \rightarrow \infty}{\lim} |s_{k(j)} - x_0| \ = \ \underset{k \rightarrow \infty}{\lim} |s_k - x_0| \ = \ \underset{s \in S}{\inf} |s -x_0|.
      \end{align}
      We show uniqueness by contradiction. Assume that another element $s' \in S$ also fulfills the mentioned condition, i.e.   
      \begin{align}
       |s' - x_0|  \ = \ \underset{s \in S}{\inf} |s' - x_0|. 
      \end{align}
      Then $|s' - x_0| = |s_0 - x_0|$, and $x_0, s_0, s'$ form an isosceles triangle. But then the midpoint of $\overline{s_0s')}$ (which is by convexity of $S$ also a member of $S$), is closer to $x_0$ than $s_0$, a 
      contradiction! \\
      As a second step, we construct a hyperplane which separates $s_0$ and $x_0$. We set
      \begin{align}
       \xi := x_0 - s_0, \ \text{and} \hspace{.5cm} \hat{\xi} := s_0 + \tfrac{1}{2} \xi.
      \end{align}
      We have
      \begin{align}
       k \ := \ \braket{\xi,\hat{\xi}} \ = \ \braket{x_0 - s_0, \tfrac{1}{2}(x_0 + s_0)} \ = \ \tfrac{1}{2}(|x_0|^2 - |s_0|^2).
      \end{align}
      By construction, $H(\xi, k)$, the hyperplane perpendicular to $\xi$, going through $\hat{\xi}$, fulfills
      \begin{align}
       0 \ < \ \tfrac{1}{2} |s_0 - x_0|^2 = \frac{1}{2}(\braket{\xi,s_0} - \braket{\xi,x_0})
      \end{align}
      Therefore, $H(\xi, k)$ is a separating Hyperplane for $s_0$ and $x_0$, since
      \begin{align}
       \braket{\xi, x_0} \ < \frac{1}{2}(\braket{\xi,s_0} - \braket{\xi,x_0}) \ < \ \braket{\xi, s_0}.
      \end{align}
      In the last two steps of the proof, we convince ourselves, that $H(\xi, k)$ is indeed separating $x_0$ and $S$. \\
      Step three. We show $H(\xi,k) \cap S = \emptyset$. Assume that $s_1 \in H(\xi,k) \cap S$, i.e. $\braket{\xi,s_1} = k$. Consider the isosceles triangle with vertices $s_1,x_0,s_0$. Let $s_2$ be a point on
      $\overline{s_0s_1}$ for which $\overline{xs_2} \perp \overline{s_0s_1}$. Then 
      \begin{align}
       |x_0 - s_2| \ < \ |x_0-s_0|,
      \end{align}
      which is a contradiction, because $s_2 \in S$. consequently the constructed hyperplane and $S$ do not intersect. \\
      Fourth step. We show, again by contradiction, that $\braket{\xi, s} > k$ for all $s \in S$, i.e. $H(\xi, k)$ separates $S$ from $x_0$. Assume hat $s_1 \in S$ and $\alpha_1 := \braket{\xi,s_1} \leq k$. 
      We already know from Step 2, that $\alpha_0 := \braket{\xi,s_0} > k$. Define
      \begin{align}
       \lambda \ := \ \frac{k - \alpha_1}{\alpha_0 - \alpha_1}.
      \end{align}
      Notice, that $\lambda$ is in $[0,1)$, since
      \begin{align}
       \alpha_0 \ > \ k \ \geq \alpha_1.
      \end{align}
      By convexity of $S$, 
      \begin{align}
       s_2 := \lambda s_0 + (1-\lambda) s_1 \  \in \ S,
      \end{align}
      which is a contradiction, because
     \begin{align}
      \braket{\xi,s_2} = \lambda \alpha_0 + (1-\lambda) \alpha_1 = k.
     \end{align}      
     \end{proof}
     Although we formulated the separating hyperplane Theorem for $\bbmR^m$ and euclidean scalar product $\braket{\cdot,\cdot}$, there is no obstacle in using it for the set $\cL(\cH)$ equipped with the Hilbert scalar 
     product $\braket{\cdot,\cdot}_{HS}$ on that space. Indeed $\Gamma: \ \bbmR^{2m^2} \ \rightarrow \ \cL(\cH)$ with
     \begin{align}
      A \ \mapsto (\Re(a_{11}),\dots, \Re(a_{mm}),\Im(a_{11}), \dots, \Im(a_{mm}))
     \end{align}
     sets up a linear isomorphism between $\bbmR^{2m^2}$ and $\cL(\cH)$, and it is easily checked, that
     \begin{align}
      \braket{A,B}_{HS} \ = \ \braket{\Gamma(A), \Gamma(B)} 
     \end{align}
     for all $A,B \in \cL(\cH)$. 
     \begin{theorem}
      A state $\rho \in \cS(\cH_A \otimes \cH_B)$ is
     \begin{itemize}
      \item separable, if and only if $\tr\rho A \geq 0$ for each entanglement witness $A$. 
      \item entangled, if and only if there exists an entanglement witness $A$, such that $\tr \rho A < 0$.
     \end{itemize}
     \end{theorem}
     \begin{proof}
      The both claims of the above theorem are easily seen to be equivalent. We prove the first claim. Asssume, that $\rho$ is separable. By defnition, $\tr \rho A \geq 0$ for each entanglement 
      witness $A$. for the converse statement, we notice, that the set of separable states on $\cS(\cH_A \otimes \cH_B)$ is a closed convex subset of $\cL(\cH)$. By Theorem \ref{thm:separating_hyperplanes} and
      the remark after its proof, we find a Hyperplane $H(A,k) \subset \cL(\cH)$, such that $\braket{A,\sigma}_{HS} \geq k$ for each member $\sigma$ of the separable density matrix, and $\braket{A,\rho} < k$.
      Consequently, $A - k \bbmeins_{\cH_A \otimes \cH_B}$ is an entanglement witness for $\rho$.
     \end{proof}
    \end{section}
   \begin{section}{Exercises}
     \begin{exercise}[Non-cyclicity of the partial trace]
     	Find an example of matrices $A,B \in \cL(\cH_1 \otimes \cH_2)$ such that 
     	$\tr_{\cH_2}(AB) \ \neq \ \tr_{\cH_2}(BA)$ holds.
     \end{exercise}
     
   \begin{exercise}[Compatibility problems]
   Let $q \in \cP(\cX_1 \times \cX_2)$, $r \in \cP(\cX_2 \times \cX_3)$ be probability distributions. They are called \emph{compatible}, if there is a probability distribution $p \in \cP(\cX_1 \times \cX_2 \times \cX_3)$ 
   having $q$ and $r$ as marginals on $\cX_1 \times \cX_2$ and $\cX_2 \times \cX_3$ respectively. 
   \begin{itemize}
    \item Show, that $q, r$ are compatible if and only if their marginals on the ``shared alphabet'' $\cX_2$ coincide.
    \item Show, by counterexample, that an analogue of the above equivalence does not hold for density matrices. (Hint: Monogamy of entanglement).
   \end{itemize}
   \end{exercise}
   \begin{exercise}
    Mixing may cause destruction of entanglement. Convince yourself by straightforward calculation, that the equidistributed mixture 
    \begin{align}
     \rho := \frac{1}{2}(\ket{\Phi_+}\bra{\Phi_+} + \ket{\Phi_-}\bra{\Phi_-})
    \end{align}
   of the so-called singlet states with state vectors $\Phi_+$ and $\Phi_-$,
   \begin{align}
    \Phi_{\pm} := \frac{1}{\sqrt{2}}( e_0 \otimes e_0 \pm e_1 \otimes e_1 ) 
   \end{align}
   is in fact separable.
   \end{exercise}
   \begin{exercise}
    The set of separable density matrices in $\cS(\cH_A \otimes \cH_B)$ is convex by definition. What are the extremal elements?
   \end{exercise}
	\begin{exercise}\label{ex:schmidt_dec}
		Let $\rho \in \cS(\cH)$ be a quantum state. What is the minimum dimension of a Hilbert space $\cK$ such that we find a purification of $\rho$ on $\cS(\cH \otimes \cK)$?
    \end{exercise}
    \begin{exercise}
	  Let $\rho \in \cS(\cH)$ be a density matrix with spectral decomposition $\sum_{i=1}^r \lambda_i \ket{f_i}\bra{f_i}$. Show, that with a pure maximally entangled state $\Phi := \ket{\phi}\bra{\phi}$ with state vector
	  \begin{align*}
	   \phi \ := \ \frac{1}{\sqrt{r}} \sum_{i=1}^{r} f_i \otimes f_i
	  \end{align*} 
	  on $\cH \otimes \cH$, 
	  \begin{align*}
		 \psi := (\bbmeins_{\cH} \otimes \rho^{\frac{1}{2}})\phi 
	  \end{align*}
	  is state vector of a purification of $\rho$.
    \end{exercise}
	\begin{exercise}[Classical purifications] \label{ex:class_pur}
	Let $p \in \cP(\cX)$ be mixed. There exists no $\cY$ such that $r \in \cP(\cX \times \cY)$ is pure and the $\cX$-marginal is $p$.
	\end{exercise}
  %   \margpar{Include \begin{itemize}
  %   		\item Exercises on Block matrices
  %   		\item Positivy preservation under isomorphisms
   %  		\item GHZ state
  %   	\end{itemize}}
   \end{section}
   
    

 
 
 
  
 
 
   
