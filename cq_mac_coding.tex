In earlier sections it was already considered a situation, where a sender aims to transmit classical messages to a receiver via a memoryless classical-quantum channel. In this section we extend the communication scenario to include a second sender. \newline 
We assume presence of a cq channel 
\begin{align}
 W: \cX \times \cY \ \rightarrow \cS(\cH)
\end{align}
The \emph{discrete memoryless classical-quantum multiple-access channel} generated by $W$ is the channel model, where the transmission is governed for each blocklength $n$ by the cq channel
\begin{align}
 W^{\otimes n}(x, y) \ := \ \bigotimes_{i=1}^n \ W(y_i, x_i)
\end{align}
for each $x = (x_1,\dots,x_n) \in \cX^n$ and $y = (y_1,\dots,y_n) \in \cY^n$. 
\begin{example}
independent example
\end{example} 
\begin{example}
dependent example
\end{example} 
As demonstrated by the second example above, the inputs of on sender can disturb the other senders signal considerably. 
Below, there is a schematic picture of the general coding scenario in case of a number of $n$ uses of the channel. 
\begin{definition}
An $(n, M_1, M_2)$ \emph{message transmission code} for the MAC $W$ is a family $\cC(u_{1,m_1}, u_{2,m_2} D_{m_1,m_2})_{m_1=1, m_2=1}^{M_1,M_2}$, where $u_{1,1},\dots,u_{1,M_1} \in \cX^n$, $u_{2,1},\dots,u_{2,M_2} \in \cY^n$, and $(D_{m_1,m_2})_{m_1=1,m_2=1}^{M_1,M_2}$ is a POVM on $\cS(\cH)$.
The \emph{(average) transmission error} of $\cC$ defined by 
\begin{align}
 \overline{e}(\cC, W^{\otimes n}) \ := \ \frac{1}{M_1 \cdot M_2} \sum_{m_1=1}^{M_1} \sum_{m_2=1}^{M_2} \tr D_{m_1,m_2}^c W^{\otimes n}(m_1,m_2).
\end{align}
We remember the notation $A^c := \bbmeins - A$. 
\end{definition} 


\begin{definition}
A pair $(R_1,R_2)$ of nonnegative numbers is called an \emph{achievable rate pair} for message transmission over the MAC $W$, if there is a sequence $(\cC_n)_{n \in \bbmN}$, each $\cC_n$ being an $(n,M_{1,n}, M_{2,n})$-code such that 
\begin{enumerate}
	\item $\lim_{n \rightarrow \infty} \ \overline{e}(\cC_n, W^{\otimes n}) \ = \ 0$,
	\item $\liminf_{n \rightarrow \infty} \ \frac{1}{n} \log M_{1,n} \ \geq R_1$, and
	\item $\liminf_{n \rightarrow \infty} \ \frac{1}{n} \log M_{2,n} \ \geq R_2$
\end{enumerate}
We define the \emph{message transmission capacity region} of the MAC $W$ by 
\begin{align*}
 \overline{C}_{MAC}(W) \ := \ \{(R_1,R_2):\ (R_1,R_2) \ \text{achievable message transmission rate pair} \}
\end{align*}
\end{definition}
The assertions in the following proposition are direct consequences from the definitions made. 
\begin{proposition}
For each cq MAC $W$, the capacity region $\overline{C}_{MAC}(W)$ is a compact and convex subset of $\bbmR^2$.
\end{proposition}

\begin{theorem}\label{cq_mac_coding:coding_thm}
Let $W: \cX \times \cY \ \rightarrow \cS(\cH)$ be a cq channel. It holds
\begin{align}
\overline{C}_{MAC}(W) \ = \bigcup_{\substack{p \in \cP(\cX)\\q \in \cP(\cY)}} \cR(p,q),
\end{align}
\end{theorem}
we will prove Theorem \ref{cq_mac_coding:coding_thm} in two portions. Proposition ... formulates the coding theorem, while the converse is Proposition \label{cq_mac_coding:coding_thm}
\begin{proposition}\label{cq_mac_coding:coding_thm}
Let $W: \cX \times \cY \ \rightarrow \cS(\cH)$ be a cq channel. It holds
\begin{align}
\overline{C}_{MAC}(W) \ \supset \bigcup_{\substack{p \in \cP(\cX)\\q \in \cP(\cY)}} \cR(p,q),
\end{align}
\end{proposition}

\begin{proposition}
Let $W: \cX \times \cY \ \rightarrow \cS(\cH)$ be a cq channel. It holds
\begin{align}
\overline{C}_{MAC}(W) \ \subset \bigcup_{\substack{p \in \cP(\cX)\\q \in \cP(\cY)}} \cR(p,q),
\end{align}
\end{proposition}
\begin{proof}
Let $p \in \cP(\cX), q \in \cP(\cY)$ be arbitrary and fixed probability distributions. We show the inclusion $\overline{C}_{MAC}(W) \subset \cR(p,q)$. \newline 
Fix an arbitrary number $\delta > 0$. We show achievability for the rate pair $(R_1, R_2)$,
\begin{align}
 R_1 \ &:= \ I(A;C) - \delta  \\
 R_2 \ &:= \ I(B;C|A) - \delta
\end{align} 
Fix $n \in \bbmN$. We will randomly choose codewords for each sender and use the random coding lemma ... . 
We introduce two independent families of i.i.d. random variables 
\begin{align}
 U &:= (U_1,\dots, U_{M_1}) \; \text{with} \ \prob(U_i = x^n) = p^n(x^n) \; \text{for each} \ x^n \in \cX^n \\
 V &:= (V_1,\dots, V_{M_2}) \; \text{with} \ \prob(V_{i} = y^n) = q^n(y^n) \; \text{for each} \ y^n \in \cY^n. \\
\end{align}
We define for each realization $v$ of $V$ a cq channel
\begin{align}
W_{1,v}: \ \cX^n \ \rightarrow \cS(\cH^{\otimes n}), \; x \mapsto W_{1,y}(x^n) := \frac{1}{M_2} \sum_{m_2=1}^{M_2} W^{\otimes n}(x^n, v_{m_2})
\end{align}
and for each realization $v$ of $V$ a cq channel 
\begin{align}
W_{2,u}: \ \cY^n \ \rightarrow \cS(\cH^{\otimes n}), \; y \mapsto W_{2,x}(y^n) := \frac{1}{M_1}\sum_{m_1=1}^{M_1}  \ket{u_{m_1}}\bra{u_{m_1}} \otimes  W^{\otimes n}(u_{m_1}, y^n)
\end{align}


\end{proof}


